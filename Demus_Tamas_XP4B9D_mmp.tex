\documentclass[10pt, final]{article}

% Set up page layout
\usepackage[a4paper, margin=1in]{geometry}
% \usepackage{indentfirst}

% Language support
\usepackage{t1enc}
\usepackage[english]{babel}

% Bibliography, refs, citations
\usepackage[sorting=none, backend=biber, style=numeric]{biblatex}
\addbibresource{mmp.bib}
\usepackage{hyperref}
\hypersetup{
	colorlinks=true,
}

% Itemize with letters
\usepackage{enumitem}

% Special math representation (matrix, etc.)
\usepackage{amsmath}
\usepackage{amsfonts}

% Software code
\usepackage{listings}

% Graphics
\usepackage{graphicx}
\usepackage{tikz}
\usepackage{pgfplots}
\pgfplotsset{compat=1.12}
\usepackage{forest}

% Development tools - only if draft set in the documentclass options
\usepackage{ifdraft}
\ifdraft {\usepackage{showframe}}


\title{Railway Track Fault Detection\\\large Mathemathical Modeling Practice\\}
\author{Tamás DEMUS\\XP4B9D}
\date{Fall Semester 2022}

\begin{document}
	\maketitle
	\tableofcontents
	\section{Introduction}
	Analyzing images and processing the information stored within 
	is a key field of machine learning.
	Classification of different images, identifying and localizing 
	different objects are basic problems.
	Several real-life cases prove the usability of such approach 
	such as traffic sign recognition, object detection or face recognition.
	The target of current research is to endeavour to create an algorithm
	that is able to classify images taken from parts of the rail track
	 to classify whether the rail is defect or not.
	\section{Dataset description}
	The dataset used for this study is taken from Kaggle webpage \cite*{noauthor_kaggle_nodate}
	and can be downloaded directly from 
	\url{https://www.kaggle.com/datasets/salmaneunus/railway-track-fault-detection}
	\cite*{noauthor_railway_nodate}.
	The dataset is stored in different directories related to their purpose: Train, Validation,
	or Test dataset.
	Inside each directory the classes also splitted to separate directories: Defective or Non defective.
	The directory structure along with the number of images can be seen in Table \ref{table:dir_struct}
	\begin{table}[!h]
		\centering
		\begin{tabular}{l c}
			Folder & Number of images \\
			\hline
			./Train/Defective & 150 \\
			./Train/Non defective & 150 \\
			./Validation/Defective & 31 \\
			./Validation/Non defective & 31 \\
			./Test/Defective & 11 \\
			./Test/Non defective & 11 \\
			\hline
		\end{tabular}
		\caption{Dataset directory structure}
		\label{table:dir_struct}
	\end{table}
	The images are taken from different perspectives (side, top), from different distances (close, distant)
	and from different parts (single rail, complete rail track, etc.) of the railway track. 
	Some examples are shown in Figure \ref{fig:track_example}.
	\begin{figure}[!h]
		

		\caption{Example images of the track}
		\label{fig:track_example}
	\end{figure}

	\section{Problem statement}
	\section{Methodology}
	\subsection{Data exploration}
	\subsection{Image processing}
	\subsection{Neural network}
	\section{Results}
	\subsection{Data exploration}
	\subsection{Image processing}
	\subsection{Neural network}
	\section{Conclusion}
	\listoffigures
	\listoftables
	\printbibliography
\end{document}