\documentclass[aspectratio=169]{beamer}
\usepackage{./tex_refs/style_pres}

\useoutertheme{infolines}
\setbeamertemplate{navigation symbols}{}

\title{Railway Track Fault Detection}
\subtitle{Mathemathical Modeling Practice}
\author{Tamás DEMUS (XP4B9D)}
\date{Fall Semester 2022}

\begin{document}
\maketitle

\begin{frame}{Table of Contents}
    \tableofcontents
\end{frame}

\section{Dataset introduction and Problem Statement}
\begin{frame}{Dataset introduction and Problem Statement}
    \begin{columns}[c]
        \column{.45\textwidth}
        \begin{exampleblock}{Example images}
            \begin{columns}[c]
                \column{.45\textwidth}
                \includegraphics[height=2.2cm]{./data/Train/Non defective/1.jpg}
                \includegraphics[height=2.2cm]{./data/Train/Non defective/22.jpg}
                \centering
                Non defective
                \column{.45\textwidth}
                \includegraphics[height=2.2cm]{./data/Train/Defective/152.jpg}
                \includegraphics[height=2.2cm]{./data/Train/Defective/220.jpg}
                \centering
                Defective
            \end{columns}
        \end{exampleblock}
        \pause
        \column{.45\textwidth}
        \begin{block}{}
            \begin{table}[!ht]
                \begin{tabular}{l c}
                    Dataset type & Number of images \\
                    \hline
                    Training     & 2x150            \\
                    Validation   & 2x31             \\
                    Test         & 2x11             \\
                \end{tabular}
            \end{table}
            \pause
            \begin{enumerate}[label=Q\arabic*]
                \item \label{itm:Q1} What kind of defects are represented in the images?
                \item \label{itm:Q2} Can these defects detected by applying image manipulation
                      and machine learning approach?
                \item \label{itm:Q3} What accuracy rate can be achieved with the algorithm?
            \end{enumerate}
        \end{block}
    \end{columns}
\end{frame}

\begin{frame}{Defect types}
    \begin{columns}[b]
        \column{.30\textwidth}
        \centering
        \includegraphics[width=0.8\textwidth]{./data/Train/Defective/170.jpg}
        Cracked rail
        \column{.30\textwidth}
        \centering
        \includegraphics[width=0.8\textwidth]{./data/Train/Defective/181.jpg}
        Disjoint rails
        \column{.30\textwidth}
        \centering
        \includegraphics[width=0.8\textwidth]{./data/Train/Defective/260.jpg}
        Surface pitting
    \end{columns}
    \begin{columns}[c]
        \column{.30\textwidth}
        \centering
        \includegraphics[width=0.8\textwidth]{./data/Train/Defective/230.jpg}
        Missing spring
        \column{.30\textwidth}
        \centering
        \includegraphics[width=0.8\textwidth]{./data/Train/Defective/190.jpg}
        Missing fastener
    \end{columns}
\end{frame}

\section{Convolutional Neural Networks}
\begin{frame}{Convolutional Neural Networks}
    \begin{columns}[T]
        \column{0.4\textwidth}
        \begin{block}{Timeline}
            \begin{tabular}{r |@{\hspace{-2.7pt}$\bullet$ \hspace{5pt}} l}
                1989 & ConvNet              \\
                1998 & {\color{red}LeNet}   \\
                2012 & {\color{red}AlexNet} \\
                     & GoogleNet            \\
                2014 & Inception            \\
                     & {\color{red}VGG}     \\
                2015 & {\color{red}ResNet}  \\
                2016 & DenseNet             \\
                2017 & ResNeXt              \\
                2018 & Channel Boosted CNN  \\
                2019 & EfficientNet         \\
            \end{tabular}
        \end{block}
        \column{0.55\textwidth}
        \begin{block}{Settings}
            \begin{tabular}{l l}
                Optimizer         & Adam                    \\
                Loss function     & Binary crossentropy     \\
                Learning rate     & Manually tuned          \\
                Callbacks         & ModelCheckPoint         \\
                                  & EarlyStopping           \\
                                  & ReduceLROnPlateau       \\
                Data augmentation & Separated from pipeline \\
                                  & 2x25 images             \\
                                  & Rotation, Zoom          \\
            \end{tabular}
        \end{block}
    \end{columns}
\end{frame}

\section{Results}
\begin{frame}{Results}
    \begin{columns}[T]
        \column{0.45\textwidth}
        \begin{block}{LeNet-5}
            \includegraphics[width=\textwidth]{./tex_graphs/metrics_LeNet-5}
        \end{block}
        \pause
        \column{0.45\textwidth}
        \begin{block}{AlexNet}
            \includegraphics[width=\textwidth]{./tex_graphs/metrics_AlexNet}
        \end{block}
    \end{columns}
\end{frame}
\begin{frame}{Results}
    \begin{columns}[T]
        \column{0.45\textwidth}
        \begin{block}{VGG16}
            \includegraphics[width=\textwidth]{./tex_graphs/metrics_VGG16}
        \end{block}
        \pause
        \column{0.45\textwidth}
        \begin{block}{Pretrained VGG16}
            \includegraphics[width=\textwidth]{./tex_graphs/metrics_VGG16_pretrained.png}
        \end{block}
    \end{columns}
\end{frame}
\begin{frame}{Results}
    \begin{columns}[T]
        \column{0.45\textwidth}
        \begin{block}{Pretrained ResNet50}
            \includegraphics[width=\textwidth]{./tex_graphs/metrics_ResNet50_pretrained}
        \end{block}
        \pause
        \column{0.45\textwidth}
        \begin{block}{Fine-tuned ResNet50}
            \includegraphics[width=\textwidth]{./tex_graphs/metrics_ResNet50_pretrained_finetuned.png}
        \end{block}
    \end{columns}
\end{frame}

\section{Hypertuning and bootstrapping}
\begin{frame}{}
    \begin{columns}
        \column{0.45\textwidth}
        \begin{block}{Hypertuning}
            \begin{itemize}
                \item Find best fit on validation dataset
                \item RandomSearch on Learning Rate
            \end{itemize}
        \end{block}
        \pause
        \column{0.45\textwidth}
        \begin{block}{Bootstrapping}
            \begin{itemize}
                \item Mitigate test dataset representativity
                \item 10 iterations with best LR
            \end{itemize}
        \end{block}
        \pause
    \end{columns}
    \centering
    \includegraphics[height=5.5cm]{./tex_graphs/bootstrap_results.png}
\end{frame}

\section{Conclusion and further steps}
\begin{frame}{}
    \begin{columns}[T]
        \column{0.45\textwidth}
        \begin{block}{Conclusion}
            \begin{itemize}
                \item AlexNet and VGG learned the training and validation sets
                \item No generalization of the model to the test dataset
                \item Possible overfitting on training and validation data
                \item Test dataset representativeness
            \end{itemize}
        \end{block}
        \pause
        \column{0.45\textwidth}
        \begin{block}{Further steps}
            \begin{itemize}
                \item Hypertuning further parameters
                \item Data augmentation in pipeline
                \item Weight initialization
                \item Additional models: VGG19, ResNet34
                \item ResNet fine-tuning
            \end{itemize}
        \end{block}
    \end{columns}
\end{frame}

\begin{frame}
    \centering \Large
    Thank you very much for your kind attention!
\end{frame}
\end{document}